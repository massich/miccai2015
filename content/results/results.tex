% include the figures path relative to the master file
\graphicspath{ {./content/results/figures/} }
\input{./content/results/figures/quantitative_figure_definitions.tex}

\begin{figure}[t]
  \centering

  \begin{tikzpicture}
  \tikzstyle{noMargin} = [inner sep=0mm, outer sep=0mm]
  \node[noMargin](a){
    \includegraphics[trim= 20 0 30 0, clip, height=3cm]{goodQSorigin}
  };
  \node[noMargin, right= 3pt of a](b){
    \includegraphics[trim= 20 0 30 0, clip, height=3cm]{goodQSseg}
    };
  \node[noMargin, right= 3pt of b](c){
    \includegraphics[trim = 0 90 0 0, clip, height=3cm]{fporigin}
  };
  \node[noMargin, right= 3pt of c]{
    \begin{tikzpicture}
      \node[noMargin](d){
      \includegraphics[trim = 0 90 0 0, clip, height=1.2cm]{fpnohom}
      };
      \node[noMargin, below= 5pt of d]{
      \includegraphics[trim = 0 90 0 0, clip, height=1.2cm]{fpHom}
      };
    \end{tikzpicture}
  };

  \end{tikzpicture}

  \caption{\footnotesize Qualitative results.
    (a) Example 1: orignal image, super-pixels' delineations and \ac{gt}.
    (b) Differences between \ac{gt} and the delination resulting from super-pixels' boundary.
    (c) Ex. 2.
    (d) weak $V(\cdot,\cdot)$
    (e) strong $V(\cdot,\cdot)$
    \vspace{-10pt}
    }
  \label{fig:results}
\end{figure}


\section{Method evaluation and comparison}

\begin{figure}[t]
  \begin{subfigure}[b]{\textwidth}
    {\tiny \input{content/results/figures/survey_table.tex}}
    %\caption{\ac{bus} images lesion segmentation strategies compiled from the bulk of the literature: reported quantitative results and methodology highlights.}
    \label{fig:surveyResults:survey}
  \end{subfigure}
  \begin{subfigure}[b]{\textwidth}
    {\tiny \input{content/results/figures/survey.tex} }
    %\caption{{\small comparison}}
    \label{fig:surveyResults:comparison}
  \end{subfigure}
  \caption{Quantitative results compilation and comparison}
  \label{fig:surveyResults}
\end{figure}

The proposed methodology is evaluated using a dataset of 16 \ac{bus} images presenting a single lesion of variable extension.
The size of the lesions ranges from under $1/100$ to over $1/5$ of the image size.
The dataset is composed of cysts, \acp{fa}, \acp{dic} and \acp{ilc}.
Every image has accompanying multi-label \ac{gt} delineating all the depicted structures.
This dataset is now publicly available at \texttt{http://visor.udg.edu/dataset/\#breast}

The lack of publicly available data and source code, limits the comparison between the
different methods.
For this study, the results published by the other authors have been collected and expressed in terms of \ac{aov} in order to share a common metric.
Further details can be found in~\cite{massich2013phd} and summarized in Fig.\,\ref{fig:surveyResults}.

Figure~\ref{fig:surveyResults} is divided into three main parts: (i) a table on the top summarizes the core stages of each study framework, (ii) a legend box on the right side informs our testing setup and, (iii) a comparison of the different metrics in a radial manner. An extra element is also represented in this radial representation: a blue swatch delimited by two blue dashed lines. The boundaries of this swatch correspond to the performance of some expert radiologists based on an inter- and intra-observer experiments carried out by Pons et al.~\cite{gerard2013}.
It is interesting to note that some methodologies outperform this swatch.
A publicly available dataset should allow a better comparison in that regard.

The results point out the inherent capabilities of \ac{ml} to cope with data scalability and variability, induce its usage in conjunction with larger datasets.
Whereas, \ac{acm} methodologies show its effectiveness to model the boundary in a natural manner.
% The ability of \ac{ml} to cope with data variability, as well as its need of data, explains why these methodologies
% have been tested, whereas the ability of \ac{acm} methods in constraining the boundary lead to slightly better results in terms of \ac{aov}.

For our proposed framework, the performance in terms of \ac{aov} lies within the state-of-the-art despite its final delineation
limited by the capacity of the super-pixels to snap the desired boundary.
Figure~\ref{fig:results} shows some qualitative results where there are limitations of labeling super-pixels when compared with hand-drawn \ac{gt}.
Figure~\ref{fig:results} also illustrates the influence of the pair-wise term.
%depicts the differences between the delineation resulting from a proper labeling of the super-pixels and the one from the \ac{gt}.

%The iconography used for the table, relates methodology stages, the technology used and if the stages have been treated separately or as a single processing.
%The figure, also arranges the information in a radial manner for direct visual comparison.



%Comparing these results, the inconvenience of unexciting public data can be obviously highlighted, since that several of these results outperform the manual delineations performed in Pons et al.~\cite{gerard2013} on the dataset here used.


% Figure~\ref{fig:results} shows qualitative results, whereas the quantitative results from the best configuration are reported as a table in Fig.\,\ref{fig:surveyResults}. %~\cite{massich2013phd}.

%Despite the fact that a \ac{fpr} of $0.4$ seems significant, based on our experiments most of the images produce no \ac{fp} lesions.
%However, images with \ac{fp} lesions, are likely to produce more several of them as shown in \cref{fig:results}c-e.
%The amount of \ac{fp} lesions can be trimmed by applying a higher cost in the pairwise therm (compare \cref{fig:resuts:smallPWterm} and \cref{fig:results:bigPWterm}.
%Nevertheless, a severe increasing of the pairwise cost also increases the \ac{fnr} since some lesions are missed due to over-smoothing. %~\cite{massich2013phd}.
%The situation of having a larger \ac{fnr} is less desirable than reducing the \ac{fpr}.
%The \ac{fnr} reported in \cref{fig:surveyResults:method} is caused by an image within the dataset that its lesion is fully contained in a single super-pixel and still around $20\%$ of this super-pixel's area is healty tissue.

% Due to the lack of publicly available data and source code, there is no manner to perform a comparison further apart of compiling the results reported in the literature.
% Based on \emph{******** et al.} the state-of-the-art methodologies found in the literature, the results are summarized in Fig.\,\ref{fig:surveyResults} and are reported in terms of \ac{aov} for comparison purposes. The references equivalence can be found in~\cite{massich2013phd}
%compiles relevant metholodologies relevant methodologies found in the literature
%The table in \Cref{fig:surveyResults:survey} complies the evaluation reported by the authors of the most relevant methodologies found in the literature. %~\cite{massich2013phd}.
%Details about the methodologies proposed in the literature can also be found in the aforesaid table.
%Specifically it is detailed the category of technique been used for detecting the lesions, segmenting it and post-process the delineations (if any).
%The studied categories are: \ac{ml}, \ac{acm} and others.
%if the those stages are treated as independent and connected in a daisy-chain fashion, or otherwise the stages are addressed in an atomic manner.


%\Cref{fig:surveyResults:comparison} renders the information present in \cref{fig:surveyResults:survey} and \cref{fig:surveyResults:method} in a visual manner compare all the results at once.
%The methodologies arranged in a radial fashion and grouped by its most representative technology category.
%In red it can be found a small, medium and large categorization of the dataset reported for testing.
%The concentric circles represent \ac{aov}.
%The blue line correspond to the \ac{aov} results reported in the literature, whereas the black line indicates the \ac{aov} our framework scored in our testing.

%\footnote{The dataset used for testing the framework here proposed corresponds to the subset of images used by Pons et al.~\cite{gerard2013} that have accompanying multi-labelled \ac{gt}.}.

%%% Local Variables:
%%% mode: latex
%%% TeX-master: "../../master.tex"
