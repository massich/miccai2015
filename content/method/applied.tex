\section{\ac{bus} images segmentation using optimization}\label{sec:methodApp}
In this section, the problem of delineating structures in \ac{bus} images is defined as an optimization problem that can be solved by applying the framework presented in Sect.\,\ref{sec:method}.
The segmentation here proposed aims at tying a label $l\in\mathcal{L}$ (\emph{i.e.} $\{\text{lesion}, \overline{\text{lesion}}\}$ or $\{ \text{chest wall}, \text{lungs}, \dots, \text{lesion} \}$) to each element of $\mathcal{S}$ by simultaneously optimizing the data and pairwise terms as illustrated in Fig.\,\ref{fig:methodTerms}.
Choices made regarding different elements:
the representation $\mathcal{S}$, the data term $D(\cdot)$, the pairwise term $V(.)$, and the optimizer choice are summarized in Table~\ref{tab:method} and justified thereafter (see Fig.\,\ref{fig:method} for reference).

$\mathcal{S}$ is considered the result from an over-segmentation of the image using Quick-shift super-pixel~\cite{achanta2012slic}.
The structures of the breast and their rendering when using a hand-held 2D \ac{us} probe are sketched in Fig.\,\ref{fig:features}a. Figure~\ref{fig:features}b illustrates the lexicon proposed by the \ac{acr}~\cite{biradsus} and used by clinicians to perform their diagnosis. Thus, our aim is to generate a set of computer vision features which is able to encode the characteristic described in the lexicon.
The selected features are as follows:

\begin{figure}[t]
    \centering
    \begin{subfigure}[b]{0.30\textwidth}
        \centering
        \begin{tikzpicture}
          \node[inner sep=0, ] (breast) {\includegraphics[width=0.4\textwidth]{breast}};
          \node[inner sep=0, below= 2pt of breast] (slice) {\includegraphics[width=0.4\textwidth]{slice}};
        \end{tikzpicture}
        %\caption{{\small Breast structure}}
        \label{fig:features:breast}
    \end{subfigure}
    \hfill
    \begin{subfigure}[b]{0.65\textwidth}
        \centering
        \includegraphics[width=\textwidth]{lexiconReworked}
        %\caption[]%
        %{Breast lesion characteristics in \ac{us} screening influencing clinical management~\cite{biradsus}}
        \label{fig:features:lexicon}
    \end{subfigure}
    \hfill
    \caption {{\footnotesize Visual reference: (a) breast structures, (b) US BI-RADS lexicon}}
    \label{fig:features}
\end{figure}

\begin{table}[b]
  \centering
  \scriptsize
  \caption{Design choices summary}
  \label{tab:method}
  \renewcommand{\arraystretch}{1.3}
  \begin{tabular}{>{\centering\bfseries}m{1in} l}
    \hline
    $\mathcal{S}$ & Quick-Shift super-pixels \\
    & Background Echotexture: encoded in Appearance and SIFT-BoW\\
    $D(\cdot)$ & Echo Pattern: encoded in Appearance, Atlas and Brightness\\% \tikz[inner sep=0mm, outer sep=0mm]{\node[inner sep=0mm, outer sep=0mm, anchor=south west] {\includegraphics[height=40pt]{featAll}};} \\
    & Acoustic Posterior: encoded in Atlas and Brightness\\
    $V(\cdot,\cdot)$ & homogeneity as Eq.\,\eqref{eq:smoothing} \\
    $\arg \min U(\cdot)$ & Graph-Cuts \\ \hline
  \end{tabular}
  % \begin{tabular}{lll}
  %  $\mathcal{S}$: Quick-Shift super-pixels &\quad &$D(\cdot)$: \\
  %  $\arg \min:$ Graph-Cuts&&\quad Features \tikz[remember picture]{\coordinate[remember picture] (featCoord) at (0,0);}\\
  %  $V(\cdot,\cdot)$: homogeneity as Eq.\,\eqref{eq:smoothing}& & \quad Construction { \acs{svm}-\acs{rbf}} \\
  % \end{tabular}
\end{table}

\begin{description}
  \item[Appearance]
    Based on the multi-labelled \ac{gt}, a \ac{mad} histogram model for every tissue label is built. The Appearance feature is computed as the $\chi^2$ distance between a histogram of $s$ and the models generated.
  \item[Atlas]
    Based on the multi-labelled \ac{gt}, an atlas is built to encode the labels likelihood based on the location of $s$.
  \item[Brightness]
    Intensity descriptors are computed based on statistics of $s$ (\emph{i.e:} mean, median, mode) and  are compared with some intensity markers of the set $\mathcal{S}$ such as the minimum intensity value, the maximum, its mean, etc.
  \item[\ac{sift}-\ac{bof}]
    $s$ is described as a histogram of visual words based on \ac{sift}~\cite{massich2014sift}. The dictionary is built with $36$ words.
\end{description}

The relationships between the lexicon and the descriptors described previously are depicted in Table~\ref{tab:method}. More precisely, we highlight the corresponding elements of the lexicon which is encoded by each feature. A choice regarding the encoding of the data term $D(\cdot)$ has to be made by using a \ac{ml} classifier. \ac{svm} classifier with \ac{rbf} kernel is selected to determine the data model during the training stage. The pairwise term in our framework was defined as in Eq.\,\eqref{eq:smoothing}. The optimization method used as a solver to minimize our cost function $U(\cdot)$ is \ac{gc}. \ac{gc}, where appropriately applied, allows to rapidly find a strong local minima guaranteeing that no other minimum with lower energies can be found~\cite{delong2012fast}. \ac{gc} is applicable if, and only if, the pairwise term favours coherent labelling configurations and penalizes labelling configurations where neighbours' labels differ such as in our case, given by Eq.\,\eqref{eq:smoothing}.
%%% Local Variables:
%%% mode: latex
%%% TeX-master: "../../master"
%%% End:
