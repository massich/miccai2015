% include the figures path relative to the master file
\graphicspath{ {./content/intro/figures/} }

\section{Introduction}
\label{sec:intro}  % \label{} allows reference to this section


Breast cancer is the second most common cancer.
In terms of mortality, breast cancer is the fifth most common cause of cancer death.
However, it is ranked as the leading cause of cancer deaths among females in both western and economically developing countries~\cite{cancerStatistics2011}.

Medical imaging contributes to its early detection through screening programs, non-invasive diagnosis, follow-up, and similar procedures.
Despite \ac{bus} imaging not being the imaging modality of reference for breast cancer screening~\cite{smith2003american}, \ac{us} imaging has more discriminative power when compared with other image modalities to visually differentiate benign from malignant solid lesions~\cite{Stavros:1995p12392}.
In this manner, \ac{us} screening is estimated to be able to reduce
between $65\sim85\%$ of unnecessary biopsies~\cite{yuan2010multimodality}, in favour of a less traumatic short-term screening follow-up using \ac{bus} images.
As the standard for assessing these \ac{bus} images, the \ac{acr} proposes the \ac{birads} lexicon for \ac{bus} images~\cite{biradsus}.
This \ac{us} \ac{birads} lexicon is a set of standard markers that characterizes the lesions encoding the visual cues found in \ac{bus} images and facilitates their analysis.
Further details regarding the \ac{us} \ac{birads} lexicon descriptors proposed by the \ac{acr}, can be found in Sect.\,\ref{sec:methodApp}, where visual cues of \ac{bus} images and breast structures are discussed to define feature descriptors.

The incorporation of \ac{us} in screening policies and the emergence of clinical standards to assess image like the \ac{us} \ac{birads} lexicon, encourage the development of \ac{cad} systems using \ac{us} to be applied to breast cancer diagnosis.
However, this clinical assessment using lexicon is not directly applicable to \ac{cad} systems.
Shortcomings like the location and explicit delineation of the lesions need to be addressed, since those tasks are intrinsically carried out by the radiologists during their visual assessment of the images to infer the lexicon representation of the lesions.
Therefore, developing accurate segmentation methodologies for breast lesions and structures is crucial to take advantage of this already validated clinical tools.

%Regardless of the clinical utility of the \ac{us} images, such image modality suffers from different inconveniences due to strong noise natural of \ac{us} imaging  and the presence of strong \ac{us} artifacts, both degrading the overall image quality~\cite{Ensminger:2008p6920} which compromise the performance of the radiologists.
%Radiologists infer health state of the patients based on visual inspection of images which by means of some screening technique (e.g.~\ac{us}) depict physical properties of the screened body.
%The radiologic diagnosis error rates are similar to those found in any other tasks requiring human visual inspection, and such errors, are subject to the quality of the images and the ability of the reader to interpret the physical properties depicted on them\cite{manning2005perception}.
%
%Therefore the interest from the medical imaging community, also for the specific case of breast lesion assessment using \ac{us} data, in developing \ac{cad} systems that provide better instrumentation to improve image interpretation, and consequently achieve better diagnosis.

This paper proposes a highly modular and flexible framework for segmenting lesions and tissues present in \ac{bus} images.
The proposal takes advantage of an energy-based strategy to perform segmentation based on discrete optimization using super-pixels and a set of novel features analogous to the elements encoded by the \ac{us} \ac{birads} lexicon~\cite{biradsus}.

%%% Local Variables:
%%% mode: latex
%%% TeX-master: "../../master.tex"
%%% End: \section{introduction}
