\documentclass{llncs}

%% Latex documents that need direct input
%  The subcaption package allows for subfloat figure within a single float.
%  This package substitutes the depregated subfigure and subfig packages
%  allowing to have subfigures within figures, or subtables within table
%  floats. Subfloats have their own caption, and an optional global
%  caption.
%  >> WARNING: some journal templates from Springer and IEETrans might not
%              be compatible with this package forcing to use the
%              deprecated packages instead.
\usepackage{subcaption}
\usepackage{subfig}

%  The following command loads a graphics package to include images
%  in the document. It may be necessary to specify a DVI driver option,
%  e.g., [dvips], but that may be inappropriate for some LaTeX
%  installations.
\usepackage[]{graphicx}

% In order to include files without having a clear page using \include*,
% the newclude package is required
\usepackage{newclude}

\usepackage{array}

% Required for acronyms
% use \acresetall to reset the acroyms counter
\usepackage{acro}

% Managing TODOES and unfinished figures
\usepackage{todonotes}

% Mathematics extra symols and commands
\usepackage{amssymb, amsmath}
\usepackage{pifont,amsfonts} % import fonts for tick and x-mark
  % define the extra symbols
  \newcommand{\cmarkgLarge}{\text{\large \color{green!60!black!80}\ding{51}}}
  \newcommand{\cmarkrLarge}{\text{\large \color{red!60!black!80}\ding{51}}}
  \newcommand{\xmarkLarge}{\text{\large \color{red!60!black!80}\ding{55}}}
  \newcommand{\cmark}{\text{\color{green!60!black!80}\ding{51}}}
  \newcommand{\xmark}{\text{\color{red!60!black!80}\ding{55}}}

%% In order to draw some graphs
\usepackage{tikz,xifthen}
\usepackage{tikz-qtree}
\usetikzlibrary{decorations.pathmorphing} % noisy shapes
\usetikzlibrary{fit}                                            % fitting shapes to coordinates
\usetikzlibrary{backgrounds}                                    % drawing the background after the foreground
\usetikzlibrary{shapes,arrows,shadows}
\usetikzlibrary{calc,decorations.pathreplacing,decorations.markings,positioning}
\usetikzlibrary{decorations.text,shapes,patterns}
\usetikzlibrary{decorations}
\usetikzlibrary{decorations.text}
\usetikzlibrary{decorations.markings}
\usetikzlibrary{shapes}
\usetikzlibrary{patterns}
\usetikzlibrary{arrows.meta}
\usepackage{pgfplots}
\pgfplotsset{compat=1.11}

%%----- To generate stand onle tikz legends
% argument #1: any options
\newenvironment{customlegend}[1][]{%
    \begingroup
    % inits/clears the lists (which might be populated from previous
    % axes):
    \csname pgfplots@init@cleared@structures\endcsname
    \pgfplotsset{#1}%
}{%
    % draws the legend:
    \csname pgfplots@createlegend\endcsname
    \endgroup
}%
% makes \addlegendimage available (typically only available within an
% axis environment):
\def\addlegendimage{\csname pgfplots@addlegendimage\endcsname}
\pgfkeys{/pgfplots/number in legend/.style={%
        /pgfplots/legend image code/.code={%
            \node at (0.295,-0.0225){#1};
        },%
    },
}
%%---- end tikz legends


% Clever cross referencing. Using cleverref, instead of writting
% figure~\ref{...} or equation~\ref{...}, only \cref{...} is required.
% The package interprates the references and introduces the figure, fig.,
% equation, eq., etc keywords. \Cref forces first letter capital.
% >> WARNING: This package needs to be loaded after hyperref, math packages,
%             etc. if used.
%             Cleveref is recomended to load late
%\usepackage{hyperref}
\usepackage{cleveref}

\usepackage{standalone}
        % contains the latex packages
\title{ An optimization approach to segment breast lesions in ultra-sound images using clinically validated visual cues }
\titlerunning{Optimization approach to BUS lesion segmentation}  % abbreviated title (for running head)
%                                     also used for the TOC unless
%                                     \toctitle is used
%
\author{
  Joan Massich\thanks{This work was partially supported by the Regional
                      Council of Burgundy FEDER grant \emph{2013-9201AAO049S02890}
                      and by the Spanish Goverment MEC grant
                      \emph{nb.TIN2012-3171-C02-01}}\inst{1}
  \and Guillaume Lema\^{i}tre\inst{1,2}
  \and Mojdeh Rastrgoo\inst{1,2}
  \and Anke Meyer-Baese\inst{3}
  \and Joan Mart\'{i}\inst{2}
  \and Fabrice M\'{eriaudeau}\inst{1}
}
%
\authorrunning{Joan Massich et al.} % abbreviated author list (for running head)
%
%
\institute{LE2I-UMR CNRS 6306, Universit\'{e} de Bourgogne, 12 rue de la Fonderie, \\
  71200 Le Creusot, France;\\
\email{joan.massich@u-bourgogne.fr}
\and
ViCOROB, Universitat de Girona,
Campus Montilivi Ed.P4, 17071 Girona, Spain.
\and
Department of Scientific Computing, Florida State University,
\\400 Dirac Science Library Tallahassee, FL 32306, US.
}
%\addtocmark{Hamiltonian Mechanics} % additional mark in the TOC
             % contains the Title and Autor info
\input{latex/filesystem/fileSetup.tex}      % contains package and variables init.
\input{content/acronym_definition.tex}      % contains the acronims

%% Select inputing only one part of the document
%\includeonly{content/intro/intro}   % the file wihtout .tex
%\includeonly{content/other/other_content}

\begin{document}
\maketitle

\begin{abstract}
\acresetall  % reset the acronyms from the title (if any)
As long as breast cancer remains the leading cause of cancer deaths among female population world wide, developing tools to assist radiologists during the diagnosis process is necessary.
However, most of the technologies developed in the imaging laboratories are rarely integrated in this assessing process, as they are based on information cues differing from those used by clinicians.
In order to grant \ac{cad} systems with these information cues when performing non-aided diagnosis, better segmentation strategies are needed to automatically produce accurate delineations of the breast structures.
This paper proposes a highly modular and flexible framework for segmenting breast tissues and lesions present in \ac{bus} images.
This framework relies on an optimization strategy and high-level descriptors designed analogously to the visual cues used by radiologists.
The methodology is comprehensively compared to other sixteen published methodologies developed for segmenting lesions in \ac{bus} images.
The proposed methodology achieves similar results than reported in the state-of-the-art.
%The achieved results state that the proposed methodology behaves accordingly to the state-of-the-art.
\end{abstract}

\keywords{Breast Ultra-Sound, BI-RADS lexicon, Optimization based Segmentation, Machine-Learning based Segmentation, Graph-Cuts}

%% Incldue the content without .tex extension
\acresetall  % reset the acronyms from the abstract
\include*{content/intro/intro}          % the file wihtout .tex
\include*{content/method/method}
\include*{content/method/applied}
%\include*{content/features/features}
\include*{content/results/results}

\section{Conclusions}
This work presents a segmentation strategy to delineate lesions in \ac{bus} images using an optimization framework that takes advantage of all the facilities available when using \ac{ml} techniques.
Despite the limitation that the final segmentation is subject to the super-pixels' boundaries, the \ac{aov} results reported here are similar to those reported by other methodologies in the literature.
A higher \ac{aov} result can be achieved by refining the delineation resulting from our proposed framework by post-processing it with an \ac{acm}. In this manner, the contour constraints could be applied to achieve a more natural delineation.

\bibliography{./content/literature_review}   %>>>> bibliography data in report.bib
\bibliographystyle{splncs03}

\end{document}
